\documentclass[cmbright,fleqn,referee]{envauth}

%%%%% AUTHORS - PLACE YOUR OWN MACROS HERE %%%%%

\def\bSig\mathbf{\Sigma}
\newcommand{\bX}{\bm{X}}
\newcommand{\bs}{\bm{s}}
\newcommand{\bZ}{\bm{Z}}
\newcommand{\ubd}{\,\bm{\mathrm{d}}}
\newcommand{\ud}{\,\mathrm{d}}
\newcommand{\VS}{V\&S}
\newcommand{\tr}{\mbox{tr}}

\received{00 Month 2012}
\revised{00 Month 2012}
\accepted{00 Month 2012}

\newtheorem{theorem}{Theorem}

\usepackage{natbib,upgreek}

%\usepackage[nolists]{endfloat}

\runninghead{Z. SHA \textit{ET AL.}}{Global Mass Framework Documentation}

\begin{document}

\title{Global Mass Framework Documentation}

\author{Zhe Sha\affil{a}\corrauth\ , Jonathan Bamber\affil{a}, William Llovel\affil{a}, Alba Martín-Español\affil{b}, Jonathan Rougier\affil{a}, Maike Schumacher\affil{a} and Andrew Zammit-Mangion\affil{c}}

\corraddr{Building, Institute, Street Address, City, Code, Country. E-mail: zhe.sha@bristol.ac.uk}

\address{\affilnum{a}Bristol Glaciology Centre, School of Geography, University of Bristol, Bristol, BS8 1SS, UK\\
\affilnum{b}Building, Institute, Street Address, City, Code, Country\\
\affilnum{c}Building, Institute, Street Address, City, Code, Country
}

\begin{abstract}
Correctly separating the sources of sea level rise (SLR) is crucial for improving future SLR predictions. Traditionally, changes in each component of the integrated signal have been tackled separately, which has often lead to inconsistencies between the sum of these components and the integral as measured by satellite altimetry. In this paper, we produce the first physically-based and data-driven solution for the complete coupled land-ocean-solid Earth system that is consistent with the full suite of observations, prior knowledge and fundamental geophysical constraints. 

...

\end{abstract}

\keywords{Bayesian hierarchical models; Gaussian Markov random fields;\\ sea-level rise; spatial-temporal statistics; stochastic partial differential equations.}
\maketitle


\section{Introduction}
\label{s:intro}


\section{Bayesian Hierarchical Modelling}
\label{s:model}
We follow the framework in \citet{b1} and define the Bayesian hierarchical model with (i) an observation layer describing the interaction between the process and instruments, (ii) a process layer containing information on the geophysical and spatial-temporal nature of the processes and (iii) a parameter layer specifying prior beliefs on unknowns. 

\subsection{The process layer}
\label{ss:process}
All the processes are measured as yearly hight change in water equivalent (mm/yr) and denoted by $\bX(\bs)$. The global sea level change depends on the volume change of water in the world's oceans and the shape of the ocean basins. The glacio-isostatic adjustment (GIA), denoted by $\bX_{GIA}$, can be used to account for the change of ocean basins ({\color{red} cite Wu et al 2010}). The volume change of ocean is mainly due to the mass change (barystatic) and the density change (steric) of water. Denote the barystatic process by $\bX_{B}$ and the steric process by $\bX_{S}$. The steric process consists of two parts: the thermosteric process $\bX_{ST}$ induced by temperature and the halosteric process $\bX_{SH}$ due to the salinity change.


When investigating locally, the sea level change can be also strongly affected by ocean circulation $\bX_{C}$ and gravity's fingerprint $\bX_{F}$ on a fine scale ({\color{red} cf Bamber and Riva 2010}). Hence, the total sea-level change $\bX_{tot} (\bs)$ at a given location $\bs$ can be decomposed as
\begin{align}\label{x_tot}
\bX_{tot} (\bs) = \bX_{SH}(\bs) + \bX_{ST}(\bs) + \bX_{B}(\bs) + \bX_{GIA}(\bs) + \bX_{C}(\bs) + \bX_F(\bs) 
\end{align}

In the following, we specify the spatial-temporal nature for each process.

\subsubsection{The steric process}
The steric process consists of halosteric process $\bX_{SH}(\bs)$ and the thermosteric process $\bX_{ST}$. How to model these two processes?
\begin{enumerate}
\item time invariant?
\item independent?
\item stationary?
\end{enumerate} 

\subsubsection{The barystatic process}
\begin{enumerate}
\item stationary spatial process?
\item time invariant?
\end{enumerate}

\subsubsection{The GIA process}
The GIA process is modelled as spatial process that does not change over the time. In this first study, we will assume it as known. 
\begin{enumerate}
\item We need to decide what error inflation to use, $20\%$?
\item Maike provide the fixed GIA value for the first study and prior information for later use?
\end{enumerate}

\subsubsection{The fine scale processes}
For regional sea level change, the ocean circulation and gravity's finger print can both modelled as fine scale variations.
\begin{align}
&\bX_{C}(\bs) = \sum_{i=1}^{n_C} \phi_i(\bs)\eta_i + \delta_{C}(\bs) \\
&\bX_{F}(\bs) = \sum_{i=1}^{n_F} \psi_i(\bs)\beta_i + \delta_{F}(\bs)
\end{align}
where $\{\phi_i\}$ and $\{\psi_i\}$ are some basis functions, $\eta_i$ and $\beta_i$ are the corresponding coefficients, and $delta_C$ and $\delta_F$ are Gaussian errors.

In this first study, we will ignore $\bX_C(\bs)$ and $\bX_F(\bs)$ since we do not have a satisfactory model to relate them with data yet. Including these two latent process will make the model under-determined. One possible solution is to smooth these two processes out by using historic data. {\color{red} \emph{My memory of using a emulator for the finger print now faded away... What data can we use to estimate $\beta_i$ and what basis functions? Green's function??}}

\subsection{The data layer}
We are going to use the following datasets:
\begin{enumerate}
\item Altimery -- Rory -- not ready
\item Argo buoys -- who -- not ready
\item GRACE -- Maike -- ready
\end{enumerate}
\subsubsection{Data processing}
Describe the data source and how we process the data.

\subsubsection{Data modelling}
The data layer is modelled by the following linear system
\begin{align}
&\bZ_{ALT}(\bs) = \bX_{tot}(\bs) + \bm{\varepsilon}_{ALT}(\bs) \\
&\bZ_{ARGO}(\bs) = \bX_{SH}(\bs) + \bm{\varepsilon}_{ARGO}(\bs)\\
&\bZ_{GRACE}(D_i) = \int_{D_i} \rho_0(\bs) \bX_{B} \ubd \bs + \int_{D_i} \rho_E(\bs) \bX_{GIA} \ubd \bs + \int_{D_i} \rho_0(\bs) \bX_{C} \ubd \bs + \bm{\varepsilon}_{GRACE}(D_i)
\end{align}
where $\bm{\varepsilon}$ are the measurement errors for each instrument (provided by the person who processed the data), $rho$ are fixed density (or density map).
\begin{enumerate}
\item I'm not entirely sure about the last equation, especially the density notation.
\end{enumerate}
\subsection{The priors layer}


\section{Parameter Estimation}
\label{s:estimate}



\section{Results}
\label{s:result}

\section{Discussion}
\label{s:discuss}

\section*{Acknowledgements}

\begin{thebibliography}{}

\bibitem[Zammit-Mangion A. et~al.(2014)]{b1}
Akash FJ, Tirky T. 1988. Proper multivariate conditional
autoregressive models for spatial data analysis. {\it Biometrics} {\bf 196}: 173.

\bibitem[Balaram et~al.(1985a)]{b2}
Balaram CA, Neeraj G, Haq HJ, Cliff PE. 1985a. {\it
NIFT\/} 2nd edition. Boca Raton, Florida: Chapman and Hall.

\bibitem[Commel(1961)]{b18}
Commel JK. 1961. {\it National Sangget Academia}, 2nd edition.
Hoboken, New Jersey: Wiley.

\bibitem[Das and Patra(1986)]{b25}
Das B, Patra HM. 1986. Using counts to simultaneously estimate
abundance and detection probabilities in a salamander community. {\it
Herpetologica} {\bf 60}: 468--478.

\bibitem[Emrat(1974)]{b14}
Emrat T. 1974. {\it Topics in Stochastic Processes.} New York: Academic Press.

\bibitem[Emrat(1985)]{b15}
Emrat T. 1985. {\it Topics in Stochastic Processes.} New York: Academic Press.

\bibitem[Goldman et~al.(1987)]{b9}
Goldman MJ, Ewin A. 1987. A comparison of smoothing
techniques for CD4 data measured with error in a time-dependent Cox
proportional hazards model. {\it Statistics in Medicine} {\bf 17}: 2061--2077.

\bibitem[Goodsman(1972)]{b8}
Goodsman RC. 1972. Testing hypotheses in the functional linear
model. {\it Scandinavian Journal of Statistics} {\bf 30}: 241--251.

\bibitem[Gooms(1972)]{b5}
Gooms RD. 1972. Testing hypotheses in the functional linear
model. {\it Scandinavian Journal of Statistics} {\bf 31}: 315--323.

\bibitem[Gooms and Wool)(1970)]{b7}
Gooms RD, Wool NJ. 1970. A comparison of smoothing
techniques for CD4 data measured with errors in a time-dependent Cox
proportional hazards model. {\it Statistics in Medicine} {\bf 17}: 2091--2099.

\bibitem[Hackel(1985)]{b10}
Hackel P. 1985. {\it Topics in Stochastic Processes.} New York: Academic Press.

\bibitem[Holloman et~al.(1979)]{b11}
Holloman PM, Tinku HA, Gerg I. 1979. A comparison of smoothing
techniques for CD4 data measured with techniques a time-dependent Cox
proportional hazards model. {\it Statistics in Medicine} {\bf 17}: 2105--2111.

\bibitem[Juman(1986)]{b12}
Juman M. 1986. Using counts to simultaneously estimate
abundance and detection probabilities in a salamander community. {\it
Herpetologica} {\bf 61}: 482--495.

\bibitem[Kartik B(1969)]{b13}
Kartik BV. 1969. Using counts to simultaneously estimate
abundance and detection probabilities in a salamander community. {\it
Herpetologica} {\bf 62}: 511--519.

\bibitem[Lenin(1932)]{b17}
Lenin DB. 1932. A comparison of smoothing
techniques for CD4 data measured with errors in a time-dependent Cox
proportional hazards model. {\it Statistics in Medicine} {\bf 17}: 2125--2135.

\bibitem[O'Rourke(1979)]{b4}
O'Rourke D. 1996. Industrial ecology: A critical review. {\it International Journal of Environment and Pollution}
{\bf 6}: 389--112.

\bibitem[Oman and Raj(1986)]{b19}
Oman FM, Raj E. 1986. {\it Topics in Stochastic Process.} New York: Academic Press.

\bibitem[Prem et~al.(1963)]{b20}
Prem GW, Christ W, Smap J, Wills JA. 1963. {\it Topics in Stochastic Processes.} New York: Academic Press.

\bibitem[Rahul(1981)]{b30}
Rahul SR. 1981. Industrial ecology: A critical review. {\it International Journal of
Environment and Pollution} {\bf 6}: 371--375.

\bibitem[Ram and Shyam(1972)]{b6}
Ram RD, Shyam S. 1972. {\it Topics in Stochastic Process.} New York: Academic Press.

\bibitem[Ravan and Hari(1983a)]{b21}
Ravan M, Hari S. 1983. Industrial ecology: A critical review.
{\it International Journal of Environment and Pollution} {\bf 6}: 389--395.

\bibitem[Ravan and Hari(1983b)]{b22}
Ravan M, Hari S. 1984. Industrial ecology: A critical
review. {\it International Journal of Environment and Pollution} {\bf 7}: 512--519.

\bibitem[Sharma et~al.(1984)]{b16}
Sharma FJ. 1984. {\it Topics in Stochastic Process.} New York: Academic Press.

\bibitem[Vim and Potter(1988)]{b23}
Vim WE, Potter HJ. 1988. A critical review. {\it International Journal of Environment and Pollution} {\bf 9}: 635--641.

\end{thebibliography}

\appendix
%%%%

%%%%%

%\section{Computation of E$_i\{\alpha_i\}$}

(This appendix was not part of the original paper by
AV~Raveendran and is included here just for illustrative
purposes. The references are not relevant to the text of the
appendix, they are references from the bibliography used to
illustrate text before and after citations.)



\end{document}

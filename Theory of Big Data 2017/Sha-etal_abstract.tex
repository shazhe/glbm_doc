%Template prepared by grzegorz ha\l aj for second AMaMeF conference, 17 Oct 2006


\documentclass[12pt]{article}
\usepackage{calc}
\usepackage{color}
\usepackage{amsfonts}
\usepackage{latexsym}
\usepackage{placeins}
\ifx\pdftexversion\undefined
  \usepackage[dvips]{graphicx}
\else
  \usepackage[pdftex]{graphicx}
\fi
\usepackage{amssymb}
\usepackage{authblk}
\usepackage{amsmath}
\usepackage[cp1250]{inputenc}
\usepackage[OT4]{fontenc}

\addtolength{\voffset}{-3.5cm} \addtolength{\textheight}{4cm}

\renewcommand\Authfont{\scshape\small}
\renewcommand\Affilfont{\itshape\small}
\setlength{\affilsep}{0.5em}

\newcommand{\smalllineskip}{\baselineskip=15pt}
\newcommand{\keywords}[1]{{\footnotesize\hspace{0.68cm}{\textit{Keywords}: }#1\par
  \vskip.7\baselineskip}}
\renewenvironment{abstract}[0]{\small\rm
        \begin{center}ABSTRACT
        \\ \vspace{8pt}
        \begin{minipage}{5.3in}
        \hspace{1pc}}{\end{minipage}\end{center}\vspace{-1pt}}
\newcommand{\emailaddress}[1]{\newline{\sf#1}}

\let\LaTeXtitle\title
\renewcommand{\title}[1]{\LaTeXtitle{\large\textsf{\textbf{#1}}}}

%%%TITLE
\title{Spatio-temporal modelling for global sea level change}
\date{}

%%AFFILIATIONS
\author[1]{Zhe Sha}
\author[2]{Andrew Zamit-Mangion}
\author[1]{Jonathan C. Rougier}
\author[1]{Maike Schumacher}
\author[1]{William Llovel}
\author[1]{Jonathan L. Bamber}
\affil[1]{University of Bristol, UK, zhe.sha@bristol.ac.uk} 
\affil[2]{University of Wollongong, Australia}

%%DOCUMENT
\begin{document}
\maketitle

%%PLEASE PUT YOUR ABSTRACT HERE
\begin{abstract}
Sea level rise is one of the most serious and tangible consequences of future climate change. Confidence in projections of future sea level will be dictated by our ability to correctly account for observed sea level changes from the recent past. Matching estimates of sea level rise with the components that affect it is a long standing problem in geosciences spanning multiple disciplines: oceanography, glaciology, hydrology and solid Earth physics. Traditionally, each part of the problem has been tackled separately using different data, techniques and physical understanding, which has often lead to inconsistencies between the sum of these components and the integral as measured by satellite altimetry. 

In this work we will, for the first time, tackle all components simultaneously by using a Bayesian Hierarchical Model (BHM) framework and produce a physically-based and data-driven solution for the complete coupled land-ocean-solid Earth system that is consistent with the full suite of observations, prior knowledge and fundamental geophysical constraints.  In the BHM, the latent geophysical processes are modelled as multivariate Gaussian processes that are auto-correlated in time and have Mat\'ern covariance functions for describing a wide range of spatial characteristics.


The BHM was developed and tested on Antarctica\cite{azm}, where it has been used to separate surface, ice dynamic and the glacial isostatic adjustment (GIA) signals simultaneously. However on a global scale, a single update of the entire process at a 1 degree resolution involves a covariance matrix of the size $10^6 \times 10^6$. We reduce the computational cost by (1)using a sparse covariance matrix induced by a Gaussian Markov random field approximation, (2)implementing a colouring algorithm on the process points for parallel computing and, (3)reducing the communication time by exploiting the cache memory. 

We illustrate the approach and concepts with examples from the test case and present the first results where we assess the consistency of the ICE-6G GIA model against the integral of sea surface height anomalies, ARGO-derived steric variations and GRACE-derived mass exchange.




\end{abstract}
%%THE END OF ABSTRACT

\begin{thebibliography}{99}
\small
\bibitem[1]{azm} Zammit Mangion, A., Rougier, J., Schoen, N., Lindgren, F., and Bamber, J., Multivariate spatio-temporal modelling for assessing Antarctica's present-day contribution to sea-level rise, \emph{Environmetrics}, 2015, pp. 159--177.


\end{thebibliography}
\end{document}

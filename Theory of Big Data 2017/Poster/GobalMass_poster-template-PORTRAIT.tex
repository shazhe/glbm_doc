\documentclass[portrait,a0paper,fontscale=0.29, margin = 4em, final]{baposter}

\usepackage[vlined]{algorithm2e}
\usepackage{times}
\usepackage{calc}
\usepackage{url}
\usepackage{graphicx}
\usepackage{amsmath}
\usepackage{amssymb}
\usepackage{relsize}
\usepackage{multirow}
\usepackage{booktabs}
\usepackage{multicol}
\usepackage{vwcol}
\usepackage[T1]{fontenc}
\usepackage{ae}
\usepackage[round]{natbib}

\usepackage{pst-solides3d}
\definecolor{oxfordblue}{RGB}{5,60,150}

\graphicspath{{images/}}

 %%%%%%%%%%%%%%%%%%%%%%%%%%%%%%%%%%%%%%%%%%%%%%%%%%%%%%%%%%%%%%%%%%%%%%%%%%%%%%%%
 %%%% Some math symbols used in the text
 %%%%%%%%%%%%%%%%%%%%%%%%%%%%%%%%%%%%%%%%%%%%%%%%%%%%%%%%%%%%%%%%%%%%%%%%%%%%%%%%
 % Format
 \newcommand{\RotUP}[1]{\begin{sideways}#1\end{sideways}}
\newcommand{\ud}{\,\mathrm{d}}

 %%%%%%%%%%%%%%%%%%%%%%%%%%%%%%%%%%%%%%%%%%%%%%%%%%%%%%%%%%%%%%%%%%%%%%%%%%%%%%%%
 % Multicol Settings
 %%%%%%%%%%%%%%%%%%%%%%%%%%%%%%%%%%%%%%%%%%%%%%%%%%%%%%%%%%%%%%%%%%%%%%%%%%%%%%%%
 \setlength{\columnsep}{0.7em}
 \setlength{\columnseprule}{0mm}


 %%%%%%%%%%%%%%%%%%%%%%%%%%%%%%%%%%%%%%%%%%%%%%%%%%%%%%%%%%%%%%%%%%%%%%%%%%%%%%%%
 % Save space in lists. Use this after the opening of the list
 %%%%%%%%%%%%%%%%%%%%%%%%%%%%%%%%%%%%%%%%%%%%%%%%%%%%%%%%%%%%%%%%%%%%%%%%%%%%%%%%
 \newcommand{\compresslist}{
 \setlength{\leftmargin}{0pt}
 \setlength{\itemsep}{0pt}
 \setlength{\parskip}{0pt}
 \setlength{\parsep}{0pt}
 }

%%%%%%%%%%%%%%%%%%%%%%%%%%%%%%%%%%%%%%%%%%%%%%%%%%%%%%%%%%%%%%%%%%%%%%%%%%%%%
%% Begin of Document
%%%%%%%%%%%%%%%%%%%%%%%%%%%%%%%%%%%%%%%%%%%%%%%%%%%%%%%%%%%%%%%%%%%%%%%%%%%%%
\begin{document}
%%%%%%%%%%%%%%%%%%%%%%%%%%%%%%%%%%%%%%%%%%%%%%%%%%%%%%%%%%%%%%%%%%%%%%%%%%%%%
%% Here starts the poster
%%---------------------------------------------------------------------------
%% Format it to your taste with the options
%%%%%%%%%%%%%%%%%%%%%%%%%%%%%%%%%%%%%%%%%%%%%%%%%%%%%%%%%%%%%%%%%%%%%%%%%%%%%
\begin{poster}{
 % Show grid to help with alignment
 grid=false,
 % Column spacing
 colspacing= 1em,
 columns=3,
 % Color style
 headerColorOne=oxfordblue,
 borderColor=oxfordblue,
 % Format of textbox
 boxColorOne=white,
 textborder=rectangle,
 % Format of text header
 headerborder=open,
 headershape=roundedright,
 headershade=plain,
 background=none,
 headerheight=0.1\textheight}
 {}
{}
{}
{}
%\headerbox{Box title}
%{name=title,span=3, row=0,column=0}{
%Box body
%}
\setlength{\colheight}{\textheight}%

%%%%%%%%%%%%%%%%%%%%%%%%%%%%%%%%%%%%%%%%%%%%%%%%%%%%%%%%%%%%%%%%%%%%%%%%%%%%%%
%%% Now define the boxes that make up the poster
%%%---------------------------------------------------------------------------
%%% Each box has a name and can be placed absolutely or relatively.
%%% The only inconvenience is that you can only specify a relative position
%%% towards an already declared box. So if you have a box attached to the
%%% bottom, one to the top and a third one which should be inbetween, you
%%% have to specify the top and bottom boxes before you specify the middle
%%% box.
%%%%%%%%%%%%%%%%%%%%%%%%%%%%%%%%%%%%%%%%%%%%%%%%%%%%%%%%%%%%%%%%%%%%%%%%%%%%%%

%%%%%%%%%%%%%%%%%%%%%%%%%%%%%%%%%%%%%%%%%%%%%%%%%%%%%%%%%%%%%%%%%%%%%%%%%%%%%%
  \headerbox{\color{white}{1. Motivation: A multilevel CAR model for forest restoration }}{name=intro,column=0,row=0, span=3}{
\begin{multicols}{2}  
\textbf{The Sabah biodiversity experiment}\\
Measurements of more than 7000 trees in 124 study plots are collected over 10 years and researchers want to investigate:
\begin{itemize}
\item tree growth rate $\sim$ biodiversity $+$ species?
\item any spatial correlation?
\begin{itemize}
\item tree level: $\varepsilon_{ij} | \delta(\varepsilon_{ij}) \sim \mathcal{N}(\rho_{e}\sum \delta(\varepsilon_{ij}), \sigma_e^2)$
\item plot level: $b_j | \delta(b_j) \sim \mathcal{N}(\rho_p \sum \delta(b_j), \sigma_p^2)$
\end{itemize}
\end{itemize} 
\textbf{A multilevel CAR model}
\begin{align*}
\boldsymbol{y} = \boldsymbol{X}\boldsymbol{\beta} + Z\boldsymbol{b} + \boldsymbol{\varepsilon}; \; \boldsymbol{b} \sim \mathcal{N}(\boldsymbol{0}, \Sigma_b), \; \boldsymbol{\varepsilon} \sim \mathcal{N}(\boldsymbol{0}, \Sigma_{\varepsilon})
\end{align*}
$ \Sigma_b = (I - \rho_p W_p)^{-1} \otimes diag(\Sigma_{b_j}), \; \Sigma_{b_j} = \sigma_b^2 I_{n_j}; 
\Sigma_{\varepsilon} = \sigma^2 (I - \rho_e W_e)^{-1} $

\centering
\includegraphics[width = 0.4\textwidth]{images/SBE_experiment_design_graphic.pdf}\\
\citep{Hector3303}
\end{multicols}
  }

%%%%%%%%%%%%%%%%%%%%%%%%%%%%%%%%%%%%%%%%%%%%%%%%%%%%%%%%%%%%%%%%%%%%%%%%%%%%%%
  \headerbox{\color{white}{2. The likelihood evaluation}}{name=logLik,column=0,row=0, span = 2, below=intro}{ 
 The log-likelihood function $\ell(\theta; y)$ is \\
 \vspace{0.03cm}

$ \propto \frac{\log(|\Sigma_{\varepsilon}| |\Sigma_b|)}{-2} + 
\log \!\int \!\exp \!\left\{\frac{\boldsymbol{\tilde{e}}^T \Sigma_{\varepsilon}^{-1}\boldsymbol{\tilde{e}}^T +
\boldsymbol{b}^T \Sigma_{b}^{-1} \boldsymbol{b}}{-2} \right\} \ud \boldsymbol{b} 
\propto -\frac{1}{2}\left(\log|\Sigma_y| + \boldsymbol{e}^T \Sigma_{y}^{-1}\boldsymbol{e}^T \right)
$ \\

\vspace{0.02cm}
 $\Sigma_y = Z^T \Sigma_b Z + \Sigma_{\varepsilon}, \;
\boldsymbol{\tilde{e}} = \boldsymbol{y} -  \boldsymbol{X}\boldsymbol{\beta} - Z\boldsymbol{b}, \; 
\boldsymbol{e} = \boldsymbol{y} -  \boldsymbol{X}\boldsymbol{\beta} $  \\
 $|\Sigma_y|$ and $\Sigma_{y}^{-1}$ takes \textcolor{red}{$\mathcal{O}(n^3)$} flops but $\Sigma_{\varepsilon}^{-1}$ and $\Sigma_b^{-1}$ are \textcolor{red}{\emph{sparse}}!
  }
  %%%%%%%%%%%%%%%%%%%%%%%%%%%%%%%%%%%%%%%%%%%%%%%%%%%%%%%%%%%%%%%%%%%%%%%%%%%%%%
 
  \headerbox{\color{white}{\large{3. The Monte Carlo likelihood}}}{name=MCL,column=2, below = intro}{
  For the likelihood $L(\theta; y) = \int f_{\theta}(y,b) \ud b$, generate $s$ samples of ${b^*}^{(i)} \sim f_{\psi}(b|Y = y)$. \citep{Geyer1992}\\
\textbf{MC likeihood}: \\
 $\ell_{\psi}^{s}(\theta) = \log \frac{1}{s} \sum_i^{s} \frac{f_{\theta}(y, {b^*}^{(i)})}{f_{\psi}(y, {b^*}^{(i)})}$ \\
 \textbf{MC MLE}:
  $\hat{\theta}_{\psi}^{s} = \underset{\theta \in \Theta}{\operatorname{arg \, max}} \; \hat{\ell}_{\psi}^{s}(\theta)$\\
\textbf{MC error}: depends on $d(\theta, \psi)$ and scales as $1/\sqrt{s}$.


\centering
  \includegraphics[width = 0.45\textwidth]{images/torus.pdf}
  \includegraphics[width = 0.5\textwidth]{images/ExactEstimateVar.pdf}
  }
  
  %%%%%%%%%%%%%%%%%%%%%%%%%%%%%%%%%%%%%%%%%%%%%%%%%%%%%%%%%%%%%%%%%%%%%%%%%%%%%% 
  \headerbox{\color{white}{4. MC-MLE through RSM}}{name=RSM,column=0,below=logLik, span = 2}{
  Use evaluations at \emph{design points} to fit a low-order polynomials approximation to the target function within the \emph{design region}. Search for the maximum by exploring along the \emph{steepest ascent path}. \citep{box2007response} 

\centering
   \includegraphics[scale = 0.38]{images/rsmEvo1.pdf}
    \includegraphics[scale = 0.38]{images/rsmEvo2.pdf}

  }
  %%%%%%%%%%%%%%%%%%%%%%%%%%%%%%%%%%%%%%%%%%%%%%%%%%%%%%%%%%%%%%%%%%%%%%%%%%%%%%
\headerbox{\color{white}{5. Conclusion}}{name=conclusion,column=2,below=MCL}{
\begin{itemize}
\item simulation-based likelihood inference procedure with stable implementation and acceptable computation time
\begin{itemize}
\item regardless of initial value
\item \textcolor{red}{1 CPU hour} for the SBE data
\end{itemize}
\item fully automated and easy adaptation to more complicated spatial models.
\item available in the package \textbf{mclcar}.
\end{itemize}

  }
 
  %%%%%%%%%%%%%%%%%%%%%%%%%%%%%%%%%%%%%%%%%%%%%%%%%%%%%%%%%%%%%%%%%%%%%%%%%%%%%%
\headerbox{\color{white}{References}}{name=ref,column=0,below=RSM, span = 3}{
\renewcommand{\refname}{\vspace{-0.5cm}}
\setlength{\bibsep}{0pt plus 4ex}
\bibliographystyle{apalike}
\footnotesize{\bibliography{references}}

  }
\end{poster}

\end{document}

\documentclass[a4paper,12pt]{article}
\usepackage{color}
\usepackage{amsmath} % do fancy math
\usepackage{mathtools}
\usepackage{amsfonts}
\usepackage{bm}
\usepackage{amsthm} % math theorem proof etc
\usepackage{graphicx} % import images
\usepackage{tikz} % draw images with latex
\usepackage{pgf} % go with tikz
\usepackage{subfigure}
\usepackage{caption}
\usepackage{multirow}
\usepackage{algorithm}
\usepackage{algorithmic}
\usepackage{epstopdf}
\usepackage[round]{natbib}
\usepackage{setspace}
\usepackage[top=30mm, bottom=30mm, left=35mm,right=30mm]{geometry}
\usepackage[utf8]{inputenc}
\usepackage[acronym]{glossaries}
\newcommand{\ud}{\,\mathrm{d}}
%% Compilation need terminal
%pdflatex glossaries.tex
%makeglossaries glossaries
%pdflatex glossaries.tex

\makeglossaries

\newglossaryentry{fwmodel}
{
        name=forward model,
        description={physical model, usually a (partial) differential equation system, used to
                     solve certain geophysical process}
}
\newglossaryentry{fwmodelling}
{
        name=forward modelling,
        description={physical model, usually a (partial) differential equation system, used to
                     solve certain geophysical process}
}
\newglossaryentry{footprint}
{
        name=footprint,
        description={aa}
}

\newglossaryentry{hyperpars}
{
        name=hyper parameters,
        description={parameters in defining the covariance structure of the latent process; in particular for a Gaussian process with Mat\'{e}rn
                        covariance function, the hyper parameters are the \gls{lengthscale} $\rho$ and \gls{normvar} $\sigma^2$}
}
 
\newglossaryentry{lengthscale}
{
        name=lengthscale,
        description={define and usually denoted by $\rho$}
}
 
\newglossaryentry{normvar}
{
        name=norminal variance,
        description={usually denoted by $\sigma^2$}
}

\newglossaryentry{predmean}
{
        name=predicted mean,
        description={the mean with respect to the predictive distribution}
}
\newglossaryentry{preduncert}
{
        name=predicted uncertainty,
        description={the standard deviation with respect to the predictive distribution}
}
\newglossaryentry{predist}
{
        name=predictive distribution,
        description={the posterior marginal distribution of a given location $X_i$ that integrate out the unknown parameters and $\bm{X_{-i}}$}
}



\newglossaryentry{prior}
{
        name=prior distribution,
        description={the probability distribution for the (hyper) parameters or latent processes}
}

\newglossaryentry{priorinfo}
{
        name=prior information,
        description={information for setting up the model, including the values for the parameters,
                     constraints,  etc.}
}

\newacronym{bhm}{BHM}{Bayesian hierarchical model} 
\newacronym{gia}{GIA}{glacial isostatic adjustment}
\newacronym{gp}{GP}{Gaussian process}
\newacronym{gmrf}{GMRF}{Gaussian Markov random field}
\newacronym{iid}{iid}{independent identically distributed}
\newacronym{spde}{SPDE}{stochastic partial differential equation}
\newacronym{inla}{INLA}{integrated nested Laplace approximation}
\newacronym{mcmc}{MCMC}{Markov chain Monte Carlo}
\newacronym{pdf}{pdf}{probability density function}
 
\begin{document}
 \title{The Bayesian Hierarchical model and glossary}
\author{Zhe Sha}
\maketitle

\onehalfspacing
\numberwithin{equation}{section}



\section{Forward modelling and spatial support}



\section{The Bayesian hierarchical model}
In this section, we introduce the Bayesian hierarchical model for predicting the global \acrshort{gia} process. 

We assume that the true \acrshort{gia} process is a real-valued spatial process continues on the sphere and denote it by $\bm{Y}: \mathbb{S}^2 \mapsto \mathbb{R}$. We use one of the \acrshort{gia} solution, say from one of the \emph{ice6g} models, as the prior mean of the true process and denote it by $\bm{\mu}: \mathbb{S}^2 \mapsto \mathbb{R}$. Then the residuals between the true process and \gls{fwmodel} solutions can be modelled as a stationary Gaussian process on the sphere 
\begin{align}\label{eq:GIAresid}
 \bm{X}: = \bm{Y} - \bm{\mu} \sim \mathcal{GP}(\bm{0}, \kappa(\bm{\theta}))
\end{align}
where $\kappa(\bm{\theta})$ defines the covariance function with \gls{hyperpars} $\bm{\theta}$.

In order to assess the bias and uncertainties in the \emph{ice6g} solutions, we use the GPS observations to update the \acrshort{gia} process. The GPS  data are the yearly trends of vertical movements in millimetre at the observed locations. These observations can be regarded as a linear map of the \acrshort{gia} process with measurement errors
\begin{align}\label{eq:GPSi}
Z_i = \bm{\mathcal{A}}_i\bm{Y} + \varepsilon_i, \; i = 1,\dots N.
\end{align} 
where $\bm{\mathcal{A}}_i$ is the linear operator that maps the \acrshort{gia} process to the $i^{\mbox{th}}$ GPS observation and $\varepsilon_i$ are assumed to be independent Gaussian errors $\mathcal{N}(0, e_i^2)$. In practice, $e_i^2$ can usually be estimated from raw GPS data and therefore we set them to be fixed values from \gls{priorinfo}. 

Denote the linear operator for the GPS observation vector $\bm{Z}$ by 
\begin{align*}
\bm{\mathcal{A}} = \left[\begin{array}{c}
 \bm{\mathcal{A}}_1\\ \vdots \\ \bm{\mathcal{A}}_N \end{array} \right]
\end{align*}
Then we can write equation \ref{eq:GPSi} into the vector form
\begin{align}\label{eq:GPS}
\bm{Z} = \bm{\mathcal{A}}\bm{Y} + \bm{\varepsilon} 
\end{align}

Now combining equations \ref{eq:GIAresid} and \ref{eq:GPS}, the system simplifies to
\begin{align}
\bm{\tilde{Z}} = \bm{Z} - \bm{\mathcal{A}}\bm{\mu}= \bm{\mathcal{A}}\bm{X} + \bm{\varepsilon}
\end{align}
Hence, our final model is
\begin{align}
\left\{ \begin{array}{l}
\bm{\tilde{Z}} = \bm{\mathcal{A}}\bm{X} + \bm{\varepsilon}, \; 
\bm{\varepsilon} \sim \mathcal{N} (\bm{0}, \mbox{diag}(e_1^2, e_2^2, \dots, e_N^2)) \\
\bm{X} \sim \mathcal{GP}(\bm{0}, \kappa(\bm{\theta})) \\
\bm{\theta} \sim \bm{p}(\bm{\theta})
\end{array} \right.
\end{align}
where $\bm{\pi}(\bm{\theta})$ is the \gls{prior} for the \gls{hyperpars}.

\section{GMRF Approximation}
Suppose we would like to predict the \acrshort{gia} process on a set of grid points $\bm{S} = \{s_i: i = 1,\dots, m\}$ with a given resolution. The Gaussian process model can be computationally expensive for large scale inference since the Bayesian update scales as $\mathcal{O}(m^3)$ mainly due to the inverse of a dense covariance matrix. 

At the same time, \acrlong{gmrf} is often used for modelling discrete spatial unit. The covariance structure is defined through its inverse, the precision matrix, which is usually sparse and thus has nice computational properties.

The Gaussian process with Mat\'{e}rn covariance function can be treated as solutions to a class of \acrlong{spde}s \citep{Lindgren2011} which can then be approximated by \acrshort{gmrf} using finite element methods. 

Denote by $\bm{\tilde{X}}$ the \acrshort{gmrf} approximation of $\bm{X}$ on a given triangulation of the sphere with piecewise linear basis functions $\{ \bm{\phi}_i \}_{i \in \mathbb{N}}$, then given any location $s \in \mathbb{S}^2$
\begin{align}
\bm{X}(s) \approx \bm{\phi}_i(s)^T\bm{\tilde{X}}
\end{align}
and for a given set $\bm{S}$ of locations, we have  
\begin{align}
\bm{X}(\bm{S}) \approx \bm{C}(\bm{S})\bm{\tilde{X}}
\end{align}
where the matrix $\bm{C}$ contains basis functions for all locations.

Now with the GMRF approximation, our model becomes
\begin{align}
\left\{ \begin{array}{l}
\bm{\tilde{Z}} = \bm{\mathcal{A}}\bm{C}\bm{\tilde{X}} + \bm{\varepsilon}, \; 
\bm{\varepsilon} \sim \mathcal{N} (\bm{0}, \mbox{diag}(e_1^2, e_2^2, \dots, e_N^2)) \\
\bm{\tilde{X}} \sim \mathcal{N}(\bm{0}, \bm{Q}^{-1}(\bm{\theta})) \\
\bm{\theta} \sim \bm{p}(\bm{\theta})
\end{array} \right.
\end{align}
where $\bm{Q}$ is the precision matrix of the \acrshort{gmrf} approximation.

\section{Bayesian Inference and prediction}
The Bayesian inference requires finding the posterior distributions of the hyper parameters and the latent process. The \acrfull{mcmc} can be used in general to sample from the posteriors and the \acrfull{inla} is a fast approximation method. In the following, we use $\bm{\pi}$ for a general \acrshort{pdf} and define a few terms used in our Bayesian inference.

The Bayesian inference draw conclusions from the posterior distribution 
\begin{align}
\bm{\pi}(\bm{X}, \bm{\theta} | \bm{\tilde{Z}}) = \frac{\bm{\pi}(\bm{X},\bm{\theta}, \bm{\tilde{Z}})}{\bm{\pi}(\bm{\tilde{Z}})}
=\frac{\bm{\pi}(\bm{\tilde{Z}}| \bm{X}, \bm{\theta}) \bm{\pi}(\bm{X}|\bm{\theta})\bm{\pi}(\bm{\theta})}
{\int_{\bm{\mathcal{X}}, \bm{\Theta}}\bm{\pi}(\bm{\tilde{Z}}| \bm{X}, \bm{\theta}) \bm{\pi}(\bm{X}|\bm{\theta})\bm{\pi}(\bm{\theta})\ud\bm{X} \ud\bm{\theta}}
\end{align}
This is the joint distribution of the latent process and the hyper parameters but in practice we are more interested in the posterior marginals for inference on parameters and prediction of the latent filed separately. For simplicity, we use posterior distribution for 
\begin{align}
\bm{\pi}(\bm{\theta}| \bm{\tilde{Z}}) = 
\int_{\bm{\mathcal{X}}} \bm{\pi}(\bm{X}, \bm{\theta} | \bm{\tilde{Z}}) \ud \bm{X}
\end{align}
the posterior marginal distribution for the hyper parameters. Inference on the parameters is not of crucial importance here but provides as sanity checks for the process property such as the length of spatial correlation. The aim of the study is to predict the latent process and the corresponding uncertainty on a fine resolution map; hence define the joint predictive distribution of the latent process to be 
\begin{align}
\bm{\pi}(\bm{X}| \bm{\tilde{Z}}) = 
\int_{\bm{\Theta}} \bm{\pi}(\bm{X}, \bm{\theta} | \bm{\tilde{Z}}) \ud \bm{\theta}
\end{align}
the posterior marginal distribution of the latent process that integrate out the uncertainty of the hyper parameters. In practice, we are more interested in predicting the marginal means and variances  at a given set of locations; hence we define the point-wise \gls{predist} to be 
\begin{align}
\bm{\pi}(X_i | \bm{\tilde{Z}}) = \int_{\bm{\mathcal{X}_{-i}}} \bm{\pi}(\bm{X} | \bm{\tilde{Z}}) \ud \bm{X_{-i}}
\end{align}
where $X_i$ is the latent process at location $i$ and $\bm{X_{-i}}$ are $\bm{X}$ elsewhere. The \gls{predmean} $X_i^*$ and \gls{preduncert} $\mbox{u}(X_i)$ are the expectation and standard deviation with respect to $\bm{\pi}(X_i | \bm{\tilde{Z}})$
\begin{align}
&X_i^* = \mathbb{E}(X_i| \bm{\tilde{Z}}) = 
\int_{\mathbb{R}} X_i \bm{\pi}(X_i | \bm{\tilde{Z}}) \ud X_i \\
&\mbox{u}(X_i) = \sqrt{\mbox{Var}(X_i| \tilde{Z})} = \sqrt{\int_{\mathbb{R}} (X_i - X_i^*)^2\bm{\pi}(X_i | \bm{\tilde{Z}}) \ud X_i}
\end{align}

The \acrshort{inla} method directly approximate the $\bm{\pi}(X_i | \bm{\tilde{Z}})$ and provide the predicted mean and uncertainty as summary statistics. For the \acrshort{mcmc} approach, the \gls{predist} can be approximate by the posterior sample distribution of $X_i$ and the \gls{predmean} and \gls{preduncert} by the sample mean and standard error.

\acrshort{inla} is much faster and more efficient than MCMC in providing the \gls{predist} but it provides limited information on the joint posterior distributions. The MCMC samples can be used in various ways for exploring the posteriors.



\clearpage
 
\printglossary[type=\acronymtype]
 
\printglossary
 
\bibliographystyle{abbrvnat}
\bibliography{references}

\end{document}
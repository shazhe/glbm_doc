\documentclass[a4paper,12pt]{article}
\usepackage{color}
\usepackage{amsmath} % do fancy math
\usepackage{mathtools}
\usepackage{amsfonts}
\usepackage{bm}
\usepackage{amsthm} % math theorem proof etc
\usepackage{graphicx} % import images
\usepackage{tikz} % draw images with latex
\usepackage{pgf} % go with tikz
\usepackage{subfigure}
\usepackage{caption}
\usepackage{multirow}
\usepackage{algorithm}
\usepackage{algorithmic}
\usepackage{epstopdf}
\usepackage[round]{natbib}
\usepackage{setspace}
\usepackage[top=30mm, bottom=30mm, left=35mm,right=30mm]{geometry}
\usepackage[utf8]{inputenc}
\usepackage[acronym]{glossaries}

%% Compilation need terminal
%pdflatex glossaries.tex
%makeglossaries glossaries
%pdflatex glossaries.tex

\makeglossaries 
\newglossaryentry{latex}
{
        name=latex,
        description={Is a mark up language specially suited for 
scientific documents}
}
 
\newglossaryentry{maths}
{
        name=mathematics,
        description={Mathematics is what mathematicians do}
}
 
\newglossaryentry{formula}
{
        name=formula,
        description={A mathematical expression}
}
 
\newacronym{gia}{GIA}{glacial isostatic adjustment}
 
\newacronym{bhm}{BHM}{Bayesian hierarchical model}
 
\begin{document}
 \title{The Bayesian Hierarchical model and glossary}
\author{Zhe Sha}
\maketitle

\onehalfspacing
\numberwithin{equation}{section}
\section{The Bayesian hierarchical model}
In this section, we introduce the Bayesian hierarchical model using the \acrshort{gia} process as an example. 

We assume that the true \acrshort{gia} process is a real-valued spatial process continues on the sphere and denote it by $Y: \mathbb{S}^2 \mapsto \mathbb{R}$. 





The \Gls{latex} typesetting markup language is specially suitable 
for documents that include \gls{maths}. \Glspl{formula} are 
rendered properly an easily once one gets used to the commands.
 
Given a set of numbers, there are elementary methods to compute 
its, which is abbreviated . This 
process is similar to that used for the .
 
 
\clearpage
 
\printglossary[type=\acronymtype]
 
\printglossary
 
\end{document}
\documentclass[a4paper,12pt]{article}
\usepackage{color}
\usepackage{amsmath} % do fancy math
\usepackage{mathtools}
\usepackage{amsfonts}
\usepackage{bm}
\usepackage{amsthm} % math theorem proof etc
\usepackage{graphicx} % import images
\usepackage{tikz} % draw images with latex
\usepackage{pgf} % go with tikz
\usepackage{subfigure}
\usepackage{caption}
\usepackage{multirow}
\usepackage{algorithm}
\usepackage{algorithmic}
\usepackage{epstopdf}
\usepackage[round]{natbib}
\usepackage{setspace}
\usepackage[top=30mm, bottom=30mm, left=35mm,right=30mm]{geometry}
\usepackage[utf8]{inputenc}
\usepackage[acronym]{glossaries}

%% Compilation need terminal
%pdflatex glossaries.tex
%makeglossaries glossaries
%pdflatex glossaries.tex

\makeglossaries 
\newglossaryentry{hyperpars}
{
        name=hyper parameters,
        description={parameters in defining the covariance structure of the latent process; in particular for a Gaussian process with Mat\'{e}rn
                        covariance function, the hyper parameters are the \gls{lengthscale} $\rho$ and \gls{normvar} $\sigma^2$.}
}
 
\newglossaryentry{lengthscale}
{
        name=lengthscale,
        description={define and usually denoted by $\rho$}
}
 
\newglossaryentry{normvar}
{
        name=norminal variance,
        description={usually denoted by $\sigma^2$}
}
 
\newacronym{gia}{GIA}{glacial isostatic adjustment}
 
\newacronym{bhm}{BHM}{Bayesian hierarchical model}
 
\begin{document}
 \title{The Bayesian Hierarchical model and glossary}
\author{Zhe Sha}
\maketitle

\onehalfspacing
\numberwithin{equation}{section}
\section{The Bayesian hierarchical model}
In this section, we introduce the Bayesian hierarchical model using the \acrshort{gia} process as an example. 

We assume that the true \acrshort{gia} process is a real-valued spatial process continues on the sphere and denote it by $Y: \mathbb{S}^2 \mapsto \mathbb{R}$. We use one of the \acrshort{gia} solution, say from one of the \emph{ice6g} models, as the prior mean of the true process and denote it by $\mu: \mathbb{S}^2 \mapsto \mathbb{R}$. Then the residuals $X: = Y - \mu$ can be modelled as a stationary Gaussian process on the phere $X \sim \mathcal{GP}(0, \kappa(\theta))$, where $\kappa(\theta)$ defines the covariance function with \gls{hyperpars} $\theta$.

In order to assess the bias and uncertainties in the \emph{ice6g} solution, we use the GPS observations to update the \acrshort{gia} process. The GPS   data are the yearly trend of vertical movement in millimeter at the observed locations and it can be regard as the \acrshort{gia} process with measurement errors; therefore we model the GPS data as $X_i = A_iY_i + \varepsilon_i$.



\clearpage
 
\printglossary[type=\acronymtype]
 
\printglossary
 
\end{document}